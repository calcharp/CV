%-------------------------------------------------------------------------------
%	SECTION TITLE
%-------------------------------------------------------------------------------
\cvsection{Outreach \& Professional Development}


%-------------------------------------------------------------------------------
%	CONTENT
%-------------------------------------------------------------------------------
\cvsubsection{Service and Outreach}
\begin{cvhonors}
%---------------------------------------------------------
  \cvhonor
    {Ruaha Carnivore Project Kids for Cats Program} % Event/Organization
    {Outreach Volunteer, Southeastern Louisiana University Lab School} % Position
    {} % Location
    {2019} % Date(s)
    
%---------------------------------------------------------
  \cvhonor
    {Math Science Upward Bound} % Event/Organization
    {Teaching Assistant, Southeastern Louisiana University}
    {} % Location
    {2019} % Date(s)
    
\end{cvhonors}

\cvsubsection{Development}

\begin{cvpubs}
    \cvpub{\textbf{Biogeography R Workshop} - This workshop acted as a think tank for using R to address questions in biogeography. Here, I contributed code and used my Revticulate package to do a joint FDB-DEC analysis to determine the phylogeny of Dicynodonts (a group of non-mammal therapsids). December 11-16, 2023.}
    \vspace{.25cm}
    \newline
    \cvpub{\textbf{2023 NSF HDR Ecosystem Conference} - This was a meeting of the different institutes of NSF's Harnessing the Data Revolution Ecosystem. We discussed how artificial neural networks are increasingly being used for automated scientific discovery, and how machine learning can better be adopted into our respective fields, including issues surrounding the interpretability, scalability, and reproducibilty of it in our research. I additionally presented a poster with my TraitBlender pipeline, a tool that can be used for assessing the assumptions made when using deep learning for automated character construction. October 16-18, 2023.}
    \vspace{.25cm}
    \newline
    %--------------------------------------------------------- 
    \cvpub{\textbf{Evolutionary Biology Graduate Student Workshop} - This was a week-long workshop at Mountain Lake Biological Station where we worked on out NSF-style grant writing skills and on framing our research questions in relation to big, open problems in ecology and evolutionary research. July 29 - August 5, 2023.}
    \vspace{.25cm}
    \newline
    %--------------------------------------------------------- 
    \cvpub{\textbf{Image Datapalooza 2023} - This was a 3.5 day hackathon-style workshop where we developed new biological datasets for computer-vision competitions with the Imageomics Institute. Foundational progress was made on the TraitBlender pipeline for simulating images of imagined organisms under predetermined evo-devo processes. August 14-17, 2023.}
    \vspace{.25cm}
    \newline
    %--------------------------------------------------------- 
    \cvpub{\textbf{NSF Workshop on High-Dimensional Data Visualization} - The workshop was a meeting of domain scientists and data visualization experts to help better understand how domain scientists actually use and visualize dimensional reduction methods in their work. We had useful discussions about the theoretical implications of using dimensionality reduction for knowledge discovery versus knowledge compression, and a paper is currently being written from the discussions. June 13-15, 2023.}
    \vspace{.25cm}
    \newline
    %--------------------------------------------------------- 
    \newpage
    \cvpub{\textbf{Imageomics All-Hands Meeting} - This meeting was about discussing the current projects associated with the Imageomics Institute and how the institute as a whole can move forward. This meeting helped me to better understand some of the communication problems that were happening between the biologists and computer scientists in the institute, and helped to guide my research interests for my dissertation. March 21 - 24, 2023.}
    \vspace{.25cm}
    \newline
    %--------------------------------------------------------- 
    \cvpub{\textbf{Phenoscape TraitFest} - This workshop was about using computer vision technologies and biological ontology software, such as the Phenoscape Knowledgebase, the better understand traits. During the workshop, I developed a small pipeline to rapidly annotate landmarks from images of mammal teeth: (https://github.com/calcharp/TraitFest\_ml-morph). Jan 23-26, 2023.}
\end{cvpubs}


\cvsubsection{Software}


\begin{cvpubs}
    \cvpub{\textbf{TraitBlender} - A pipeline for generating museum-specimen style images of imagined organisms that evolve under chosen evolutionary/developmental processes\\
    (https://github.com/calcharp/TraitBlender)}
    \vspace{.25cm}
    \newline
    \cvpub{\textbf{rphenoscate} - An R package for semantic-aware evolutionary analyses of anatomical traits \\
    (https://github.com/uyedaj/rphenoscate)}
    \vspace{.25cm}
    \newline
    \cvpub{\textbf{Revticulate} – A package for interacting with the Rev language via R (https://github.com/revbayes/Revticulate)}
    \vspace{.25cm}
    \newline
    \cvpub{\textbf{SISRS} - A Python-based pipeline for identifying phylogenetically informative sites from next-generation whole-genome sequencing of multiple species (https://github.com/SchwartzLabURI/SISRS)}
\end{cvpubs}
    
% \begin{small} \color{black}
% One journal I review for \\
% Another journal \\
% \end{small}

\cvsubsection{Professional Memberships}

\begin{small} \color{black}
The Society for the Study of Evolution \\
\end{small}
